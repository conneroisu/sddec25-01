\chapter{Introdução}\label{cap:intro}
A introdução de um relatório é uma parte importante do documento, pois ela apresenta o tema, justifica a relevância do assunto e estabelece o escopo do trabalho. Algumas informações que podem ser incluídas na introdução de um relatório são:
\begin{incisos}
\item Título: O título do relatório deve ser claro e descritivo para que o leitor possa entender o tema abordado.
\item Objetivo: A introdução deve apresentar o objetivo do relatório e explicar o motivo pelo qual o trabalho foi realizado. Isso ajuda a estabelecer a importância do assunto e a justificar a relevância do relatório.
\item Contexto: É importante fornecer um contexto para o assunto do relatório. Isso pode incluir informações históricas, estatísticas ou dados relevantes que ajudem o leitor a entender a importância do assunto. 
\item Escopo: A introdução deve definir o escopo do relatório, explicando quais aspectos do assunto serão abordados e quais serão deixados de lado. Isso ajuda a orientar o leitor sobre o que esperar do relatório.
\item Metodologia: Se o relatório se basear em pesquisa ou análise, é importante descrever a metodologia utilizada para obter os resultados. Isso ajuda a estabelecer a credibilidade do relatório e a demonstrar a validade dos dados apresentados.
\item Estrutura: Por fim, a introdução deve apresentar a estrutura do relatório, explicando como as informações serão organizadas e quais tópicos serão abordados em cada seção.
\end{incisos}
Lembre-se de que a introdução deve ser clara e objetiva, apresentando o tema de forma interessante e motivando o leitor a continuar a leitura do relatório.

A seguir deixamos algumas seções predefinidas que podem ser alteradas a vontade.

Vamos escrever alguns comandos, por exemplo, como reverenciar o capítulo de desenvolvimento: De acordo com o capítulo \ref{cap:desenvolvimento}, ..... Referenciando uma seção \ref{sec:motivacao}.

Começando novo parágrafo: apenas deixar um espaço em branco (linha em branco). 

Os demais  textos deste capítulo são aleatórios gerados pelo comando \verb+lipsum+.

\lipsum[1]

%----------------------------------------------------------------------
\section{Motivação}
\label{sec:motivacao}

\lipsum[2]

%----------------------------------------------------------------------
\section{Objetivos}
\label{sec:objetivos}

\lipsum[2]

%----------------------------------------------------------------------
\section{Estrutura do Trabalho}
\label{sec:estrutura}

\lipsum[1]