\chapter{Introduction}\label{chap:introduction}

\section{Problem Statement}\label{sec:problem-statement}

Handicapped individuals with underlying conditions face the critical challenge of detecting and responding to medical episodes before they occur, which can happen anytime and anywhere, posing significant risks to their safety and independence. In a broader societal context, individuals with disabilities often encounter inadequate assistive technologies that fail to proactively ensure their well-being. Current healthcare solutions are reactive, requiring human intervention after an episode occurs, which can lead to delayed response times, severe medical complications, and loss of autonomy.

This issue is particularly significant as advancements in artificial intelligence and edge computing offer new opportunities for real-time health monitoring~\cite{chen2021edge,smith2023eyetracking}. However, these technologies remain underutilized in the field of assistive mobility devices. The ability to predict and respond to medical emergencies in real time would not only enhance personal safety but also reduce the burden on caregivers and emergency medical services, improving overall healthcare efficiency.

To address this problem, our project focuses on leveraging semantic segmentation at the edge to analyze physiological indicators such as eye movement and body posture. By integrating this technology into wheelchairs, we aim to create an intelligent system that detects early warning signs of medical distress and autonomously moves the user to a safer position before a critical incident occurs. This approach bridges the gap between existing assistive technologies and the urgent need for proactive, real-time health monitoring, using established U-Net architectures for biomedical image segmentation~\cite{ronneberger2015}, ultimately empowering handicapped individuals to navigate their daily lives with greater security and independence.

\section{Intended Users}\label{sec:intended-users}

\subsection{Primary Clients}\label{subsec:primary-clients}

The primary clients of this product are individuals with mobility impairments, many of whom have underlying physiological conditions such as Cerebral Palsy, epilepsy, or cardiovascular disorders. These individuals depend on wheelchairs for mobility and face heightened risks associated with sudden medical episodes. They require a proactive safety system that detects early signs of medical distress and responds autonomously to relocate them to a safe position.

Maintaining independence is a critical priority for these individuals, as many wish to lead active lives without constant supervision. By integrating real-time monitoring and intervention features, this product empowers users by providing an added layer of security without compromising their autonomy~\cite{beauchamp2007ethics}. The benefits of such a system include a significant reduction in medical emergencies, increased confidence in navigating daily life, and an overall improved quality of life.

\subsection{Caregivers and Family Members}\label{subsec:caregivers}

Caregivers and family members form the secondary user group, as they play an essential role in ensuring the well-being of individuals with mobility impairments. Parents, guardians, and professional caregivers are often burdened with the responsibility of constant monitoring, which can be both emotionally and physically demanding. They need a reliable alert system that provides real-time updates on the user's condition, allowing them to respond appropriately without intrusive supervision.

This product alleviates some of the stress associated with caregiving by offering automated alerts and health tracking, enabling caregivers to provide support when necessary while also granting users greater independence. The ability to receive timely notifications about potential medical issues enhances caregivers' ability to act swiftly and effectively, fostering a more sustainable care model.

\subsection{Healthcare Providers and Emergency Responders}\label{subsec:healthcare}

The tertiary user group consists of healthcare providers and emergency responders, including medical professionals, therapists, and paramedics who are responsible for diagnosing, treating, and responding to medical emergencies among mobility-impaired individuals. These professionals rely on accurate, real-time health data to assess risks and make informed decisions.

An automated system capable of detecting early warning signs of medical distress and transmitting alerts to healthcare providers can significantly improve response times and patient outcomes. Additionally, the ability to integrate this data with existing healthcare monitoring systems enhances the efficiency of medical intervention. By bridging the gap between assistive mobility technology and healthcare services, this project contributes to a more data-driven approach to patient care, ultimately improving medical decision-making and emergency response capabilities.

\section{Summary}\label{sec:summary}

Each of these user groups plays a critical role in the success and impact of this project. By addressing the needs of handicapped individuals, caregivers, and healthcare professionals, this product aims to create a safer, more autonomous, and more efficient system for managing mobility and health-related challenges. The integration of real-time monitoring and autonomous intervention not only enhances the quality of life for individuals with disabilities but also eases the burden on caregivers and improves medical response strategies. In doing so, this project contributes to a broader movement toward proactive, technology-driven healthcare solutions that prioritize safety, independence, and well-being.
