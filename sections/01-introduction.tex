\chapter{Introduction}\label{chap:introduction}

\section{Problem Statement}\label{sec:problem-statement}

Individuals with mobility impairments and underlying conditions face the critical challenge of detecting and responding to medical episodes before they occur. These episodes can happen anytime and anywhere, posing significant risks to safety and independence.

Current healthcare solutions are reactive, requiring human intervention after an episode occurs. This approach can lead to delayed response times, severe medical complications, and loss of autonomy. In a broader societal context, many assistive technologies fail to proactively ensure user well-being.

This issue is particularly significant as advancements in artificial intelligence and edge computing offer new opportunities for real-time health monitoring~\cite{lakshminarayanan2023health}. However, these technologies remain underutilized in the field of assistive mobility devices. The ability to predict and respond to medical emergencies in real-time would not only enhance personal safety but also reduce the burden on caregivers and emergency medical services, improving overall healthcare efficiency.

This work leverages semantic segmentation at the edge to analyze physiological indicators such as eye movement and body posture. The validity of these indicators is well-established in the neurological literature: abnormal eyelid dynamics contain seizure-specific anomalies that distinguish ictal events from normal behavior~\cite{sedighsarvestani2012eyelid}, while blink reflex patterns and eye movements are physiologically linked to neurological state through well-characterized motor control pathways~\cite{evinger2011blinking}.

Eyelid myoclonia has been identified as a distinct ictal sign in idiopathic generalized epilepsy~\cite{stefan2007eyelid}, demonstrating that eye-related movements provide clinically meaningful indicators of seizure activity. Prior work in eye tracking for seizure detection has demonstrated that eye and head motion patterns can serve as indicators of certain seizure types~\cite{provost2022eye}.

By integrating this technology into wheelchairs, the system detects early warning signs of medical distress and autonomously moves the user to a safer position before a critical incident occurs. This approach uses established U-Net architectures for biomedical image segmentation~\cite{ronneberger2015} to bridge the gap between existing assistive technologies and the need for proactive, real-time health monitoring.

\section{Intended Users}\label{sec:intended-users}

This system serves three user groups: (1) wheelchair users with mobility impairments and conditions such as cerebral palsy, epilepsy, or cardiovascular disorders who require autonomous medical episode detection; (2) caregivers and family members who need real-time health alerts without constant supervision; and (3) healthcare providers and emergency responders who rely on accurate physiological data for rapid medical decision-making~\cite{beauchamp2007ethics}.
