\chapter{Introduction}\label{chap:introduction}

\section{Problem Statement}\label{sec:problem-statement}

Individuals with mobility impairments and underlying conditions face
the challenge of detecting and responding to medical episodes before
they occur, which can happen anytime and anywhere, posing risks to
their safety and independence. Current assistive
technologies often fail to proactively ensure user well-being.
Current healthcare solutions are reactive, requiring human
intervention after an episode occurs, which can lead to delayed
response times, severe medical complications, and loss of autonomy.

Advances in artificial intelligence and edge computing offer new opportunities
for real-time health monitoring~\cite{lakshminarayanan2023health}, yet these
technologies remain underutilized in assistive mobility devices. The ability to predict and respond to medical emergencies in
real-time would not only enhance personal safety but also reduce the
burden on caregivers and emergency medical services, improving
overall healthcare efficiency.

Our project leverages semantic segmentation at the edge to analyze
physiological indicators such as eye movement and body posture. Neurological literature establishes these indicators: abnormal eyelid
dynamics contain seizure-specific anomalies~\cite{sedighsarvestani2012eyelid},
and blink reflex patterns link to neurological state through motor control
pathways~\cite{evinger2011blinking}. Eyelid myoclonia has been
identified as a distinct ictal sign in idiopathic generalized
epilepsy~\cite{stefan2007eyelid}, demonstrating that eye-related
movements provide clinically meaningful indicators of seizure
activity. Prior work demonstrates that eye and head motion patterns serve as
seizure indicators~\cite{provost2022eye}. Integrating this technology
into wheelchairs creates an intelligent system that detects early warning
signs and autonomously repositions the user before critical incidents occur,
using established U-Net architectures for biomedical segmentation~\cite{ronneberger2015}.

\section{Intended Users}\label{sec:intended-users}

The system serves three user groups: (1) wheelchair users with conditions
such as cerebral palsy, epilepsy, or cardiovascular disorders requiring
autonomous episode detection; (2) caregivers needing real-time health alerts
without constant supervision; and (3) healthcare providers relying on accurate
physiological data for rapid decision-making~\cite{beauchamp2007ethics}.
