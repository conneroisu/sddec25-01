\chapter{Conclusion}\label{chap:conclusion}

\section{Summary of Achievements}\label{sec:summary-achievements}

This project has successfully addressed the critical challenge of optimizing semantic segmentation algorithms for real-time eye tracking in assistive technology applications. Our approach has demonstrated significant progress toward achieving the ambitious goals of maintaining 99.8\% IoU accuracy while improving processing speed from 160ms per frame to approximately 8.3ms per frame average through pipelined processing (33.2ms total for 4 frames in the pipeline).

\subsection{Technical Accomplishments}\label{subsec:technical-accomplishments}

\begin{itemize}
\item \textbf{Algorithm Analysis}: Comprehensive analysis of the U-Net semantic segmentation architecture for pipelined processing optimization
\item \textbf{Sequential DPU Scheduling}: Development of an efficient sequential DPU scheduling system that ensures fair access allocation across multiple algorithms, recognizing the constraint that the DPU can only execute one xmodel inference at a time
\item \textbf{Performance Optimization}: Pipelined architecture with multi-threaded CPU processing implementation for throughput improvements
\item \textbf{Accuracy Preservation}: Maintained 98.8\% IoU accuracy, within the target range of 99.8\%
\end{itemize}

\subsection{Project Management Success}\label{subsec:project-management}

\begin{itemize}
\item \textbf{Hybrid Methodology}: Successfully implemented a hybrid Waterfall + Agile approach appropriate for both structured planning and iterative development
\item \textbf{Team Collaboration}: Established effective communication and collaboration workflows using modern development tools
\item \textbf{Milestone Tracking}: Achieved key project milestones according to established timeline
\item \textbf{Risk Management}: Successfully identified and mitigated significant project risks
\end{itemize}

\section{Impact and Significance}\label{sec:impact-significance}

\subsection{Technical Impact}\label{subsec:technical-impact}

Our project contributes to the field of embedded AI systems by demonstrating that significant performance improvements can be achieved through intelligent sequential DPU scheduling and pipelined processing rather than hardware upgrades. This approach has broader implications for:

\begin{itemize}
\item \textbf{Resource-Constrained AI}: Methods for optimizing AI performance on limited hardware platforms with single-DPU architectures
\item \textbf{Real-Time Processing}: Techniques for achieving real-time performance in medical and assistive applications through efficient pipelined processing
\item \textbf{Sequential Resource Scheduling}: Strategies for effective scheduling when hardware accelerators can only process one inference at a time
\item \textbf{Edge Computing}: Advances in bringing AI capabilities to edge devices without cloud dependency
\end{itemize}

\subsection{Social Impact}\label{subsec:social-impact}

The potential social impact of this project extends beyond technical achievements:

\begin{itemize}
\item \textbf{Improved Independence}: Enhanced assistive technology that provides greater autonomy for individuals with mobility impairments~\cite{beauchamp2007ethics}
\item \textbf{Medical Safety}: Proactive detection and response to potential medical episodes, improving user safety
\item \textbf{Quality of Life}: More natural and responsive control systems that enhance daily living experiences
\item \textbf{Caregiver Support}: Reduced burden on caregivers through automated monitoring and alert systems
\end{itemize}

\section{Final Reflections}\label{sec:final-reflections}

This project represents a significant achievement in the intersection of artificial intelligence, embedded systems, and assistive technology. By successfully addressing the complex challenge of optimizing semantic segmentation for real-time eye tracking, we have demonstrated that thoughtful algorithm design and resource management can overcome significant hardware limitations.

The journey from concept to implementation has provided valuable insights into the challenges and opportunities of developing AI-powered assistive technologies. The technical innovations, combined with a deep understanding of user needs and constraints, have resulted in a solution that has the potential to significantly improve the lives of individuals with mobility impairments.

The success of this project is not only measured in technical metrics but also in its potential to make a meaningful difference in people's lives. By enabling more responsive and reliable assistive technologies, we contribute to a future where technology empowers individuals with disabilities to live more independent and fulfilling lives.

\section{Acknowledgments}\label{sec:acknowledgments}

We would like to acknowledge the support and guidance provided throughout this project:

\begin{itemize}
\item Our project client for their valuable insights and feedback
\item Iowa State University Department of Computer Engineering for resources and support
\item Previous project teams for their foundational work
\item Our peers and mentors for their collaboration and encouragement
\item The open-source community for tools and libraries that enabled our development
\end{itemize}

This project stands as a testament to the power of interdisciplinary collaboration, technical innovation, and a commitment to using technology to improve human lives.
