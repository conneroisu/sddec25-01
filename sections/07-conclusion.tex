\chapter{Conclusion}\label{chap:conclusion}

\section{Summary of Achievements}\label{sec:summary-achievements}

This project optimized semantic segmentation for real-time eye tracking
in assistive technology. Our approach maintains 99.8\% IoU accuracy while
improving speed from approximately 160ms to less than 10ms per frame through
pipelined processing (less than 35ms for 4 frames).

\subsection{Technical Accomplishments}\label{subsec:technical-accomplishments}

\begin{itemize}
  \item \textbf{Algorithm Analysis}: Comprehensive analysis of the
    U-Net semantic segmentation architecture for pipelined processing
    optimization
  \item \textbf{Sequential DPU Scheduling}: Development of an
    efficient sequential DPU scheduling system that ensures fair
    access allocation across multiple algorithms
  \item \textbf{Performance Optimization}: Pipelined architecture
    with multithreaded CPU processing implementation for throughput improvements
  \item \textbf{Accuracy Preservation}: Maintained approximately 98--99\% IoU
    accuracy, within the target range of 99.8\%
\end{itemize}

\subsection{Project Management Success}\label{subsec:project-management}

\begin{itemize}
  \item \textbf{Hybrid Methodology}: Successfully implemented a
    hybrid Waterfall + Agile approach appropriate for both structured
    planning and iterative development
  \item \textbf{Team Collaboration}: Established effective
    communication and collaboration workflows using modern development tools
  \item \textbf{Milestone Tracking}: Achieved key project milestones
    according to established timeline
  \item \textbf{Risk Management}: Successfully identified and
    mitigated significant project risks
\end{itemize}

\section{Impact and Significance}\label{sec:impact-significance}

\subsection{Technical Impact}\label{subsec:technical-impact}

Our project demonstrates that performance improvements can be achieved
through intelligent DPU scheduling and pipelined processing rather than
hardware upgrades. Broader implications:

\begin{itemize}
  \item \textbf{Resource-Constrained AI}: Methods for optimizing AI
    performance on limited hardware platforms with single-DPU architectures
  \item \textbf{real-time Processing}: Techniques for achieving
    real-time performance in medical and assistive applications
    through efficient pipelined processing
  \item \textbf{Sequential Resource Scheduling}: Strategies for
    effective scheduling with shared hardware accelerators
  \item \textbf{Edge Computing}: Advances in bringing AI capabilities
    to edge devices without cloud dependency
\end{itemize}

\subsection{Social Impact}\label{subsec:social-impact}

Social impact extends beyond technical achievements:

\begin{itemize}
  \item \textbf{Improved Independence}: Enhanced assistive technology
    that provides greater autonomy for individuals with mobility
    impairments~\cite{beauchamp2007ethics}
  \item \textbf{Medical Safety}: Proactive detection and response to
    potential medical episodes, improving user safety
  \item \textbf{Quality of Life}: More natural and responsive control
    systems that enhance daily living experiences
  \item \textbf{Caregiver Support}: Reduced burden on caregivers
    through automated monitoring and alert systems
\end{itemize}

\section{Final Reflections}\label{sec:final-reflections}

This project addresses the intersection of AI, embedded systems, and
assistive technology. We demonstrated that algorithm design and resource
management can overcome hardware limitations.

Development provided insights into AI-powered assistive technologies.
The solution may improve assistive systems for individuals with
mobility impairments.

More responsive assistive technologies support individuals with disabilities
in maintaining greater independence.

\section{Acknowledgments}\label{sec:acknowledgments}

Acknowledgments for support throughout this project:

\begin{itemize}
  \item Our project client for their valuable insights and feedback
  \item Iowa State University Department of Computer Engineering for
    resources and support
  \item Previous project teams for their foundational work
  \item Our peers and mentors for their collaboration and encouragement
  \item The open-source community for tools and libraries that
    enabled our development
\end{itemize}

This project demonstrates interdisciplinary collaboration in assistive
technology development.
