\chapter{Project Plan}\label{chap:project-plan}

\section{Project Management and Tracking
Procedures}\label{sec:management-procedures}

Our team adopted a hybrid Waterfall + Agile approach. This methodology
provided Waterfall's structured framework for critical path activities
and Agile's flexibility for iterative development. This approach suited
the project because:

\begin{enumerate}
  \item Our project has clearly defined phases (pipeline architecture design,
    implementation, testing) that benefited from Waterfall planning
  \item Implementing pipelined processing, DPU scheduling, and algorithm
    optimization required adaptive iterations benefiting from Agile sprints
  \item Specialized hardware (Kria Board KV260) required careful resource
    allocation and DPU scheduling
\end{enumerate}

For project tracking, the team used:

\begin{itemize}
  \item \textbf{GitHub}: Code repository for version control and collaboration.
    The client had access to track progress in real-time.
  \item \textbf{Telegram}: Communication channel with the client and previous
    team members for quick updates.
  \item \textbf{GitLab}: The team transitioned to GitLab during the project
    for enhanced tracking capabilities. GitLab provided storage for model
    weights, training run tracking, and additional features at no cost to
    the client. While we proposed that the client invite future team members
    (who signed NDAs) to the GitLab organization to keep all teams informed,
    the client opted for a more manual process of information management
    between teams.
\end{itemize}

Weekly team meetings reviewed sprint progress and addressed blockers.
Monthly client meetings ensured alignment with project goals.

\section{Task Decomposition}\label{sec:task-decomposition}

Our project optimized the U-Net algorithm by implementing a pipelined
architecture with multi-threaded CPU processing and sequential DPU execution.
The target: 60 FPS for real-time medical monitoring. Tasks identified:

\subsection{Task 1: Algorithm Performance Optimization}
\begin{itemize}
  \item Subtask 1.1: Analyze U-Net architecture for optimization opportunities
  \item Subtask 1.2: Develop performance enhancement approach
  \item Subtask 1.3: Validate that accuracy requirements (99.8\% IoU)
    are maintained
\end{itemize}

\subsection{Task 2: Implementation of Core Components}
\begin{itemize}
  \item Subtask 2.1: Implement image pre-processing using semantic segmentation
  \item Subtask 2.2: Implement eye tracking algorithm with pre-processed images
  \item Subtask 2.3: Implement blink detection algorithm
  \item Subtask 2.4: Implement DPU sharing mechanism for resource optimization
\end{itemize}

\subsection{Task 3: Thread Management}
\begin{itemize}
  \item Subtask 3.1: Implement memory sharing between threads (non-DDR)
  \item Subtask 3.2: Configure thread allocation to specific memory locations
  \item Subtask 3.3: Implement thread synchronization and
    communication~\cite{park2022thread}
  \item Subtask 3.4: Test thread operation with matrix operations
\end{itemize}

\subsection{Task 4: Multicore Processing}
\begin{itemize}
  \item Subtask 4.1: Configure build environment for optimal performance
  \item Subtask 4.2: Develop multi-core loading method for ONNX model
  \item Subtask 4.3: Implement pipelined passing of data through threads
  \item Subtask 4.4: Optimize data flow between processing units
\end{itemize}

\subsection{Task 5: Integration and Testing}
\begin{itemize}
  \item Subtask 5.1: Integrate all components into a unified system
  \item Subtask 5.2: Benchmark performance against target metrics
  \item Subtask 5.3: Identify and resolve bottlenecks
  \item Subtask 5.4: Validate accuracy of results and compare to baseline system
\end{itemize}

\subsection{Task 6: Documentation and Delivery}
\begin{itemize}
  \item Subtask 6.1: Document implementation details and architecture
  \item Subtask 6.2: Prepare user guides and technical documentation
  \item Subtask 6.3: Develop demonstration materials
  \item Subtask 6.4: Prepare final project presentation
\end{itemize}

Tasks were broken into sprint activities with members assigned by expertise.

\section{Project Milestones, Metrics, and Evaluation
Criteria}\label{sec:milestones}

Key milestones with associated metrics and evaluation criteria:

\subsection{Milestone 1: Pipeline Architecture Design}
\begin{itemize}
  \item \textbf{Completion Date}: Week 8
  \item \textbf{Metrics}: Validated pipelined architecture design
    with overlapped CPU and DPU processing
  \item \textbf{Evaluation Criteria}: Architecture design maintains
    output accuracy equivalent to original algorithm
\end{itemize}

\subsection{Milestone 2: DPU Model Deployment and Memory Optimization}
\begin{itemize}
  \item \textbf{Completion Date}: Week 12
  \item \textbf{Metrics}: Successful loading of U-Net xmodel onto DPU
    with optimized memory allocation for pipelined data flow
  \item \textbf{Evaluation Criteria}: Model loads correctly with
    optimal memory utilization (<90\% of allocated memory) supporting
    efficient buffer management
\end{itemize}

\subsection{Milestone 3: Thread Testing with Matrix Operations}
\begin{itemize}
  \item \textbf{Completion Date}: Week 16
  \item \textbf{Metrics}: Successful multi-threaded operation for
    CPU-based preprocessing and postprocessing
  \item \textbf{Evaluation Criteria}: All CPU threads operate
    correctly with proper synchronization and without memory conflicts
\end{itemize}

\subsection{Milestone 4: Pipelined Implementation of Semantic Segmentation}
\begin{itemize}
  \item \textbf{Completion Date}: Week 16
  \item \textbf{Metrics}: Functional pipelined semantic segmentation
    system with optimized throughput
  \item \textbf{Evaluation Criteria}: System achieves efficient
    pipelined processing of multiple frames with accuracy equal to or
    greater than original implementation (99.8\% accuracy)
\end{itemize}

\subsection{Milestone 5: Increased Throughput Demonstration}
\begin{itemize}
  \item \textbf{Completion Date}: Week 16
  \item \textbf{Metrics}: Processing speed of multiple frames
  \item \textbf{Evaluation Criteria}: Achieve target throughput of
    33.2\ ms for 4 frames (vs.\ current 160\ ms for 1 frame)
\end{itemize}

For each milestone, we tracked progress using these metrics:

\begin{itemize}
  \item \textbf{Processing time}: Measured in milliseconds per frame
  \item \textbf{Accuracy}: Comparison of segmentation results with
    ground truth data
  \item \textbf{Resource utilization}: CPU, memory, and DPU usage percentages
  \item \textbf{Throughput}: Frames processed per second
\end{itemize}

\section{Project Timeline and Schedule}\label{sec:timeline}

The project spanned 16 weeks with work organized into sprints:

\subsection{Key Deliverable Dates}
\begin{itemize}
  \item \textbf{Week 8}: Model optimization and pipelining proposal document
  \item \textbf{Week 12}: Thread testing results and documentation
  \item \textbf{Week 16}: Preliminary performance report
\end{itemize}

\subsection{Critical Path}
The critical path follows U-Net optimization, eye-tracking implementation,
pipeline integration, and throughput optimization.

\subsection{Sprint Organization}
\begin{itemize}
  \item \textbf{Sprints 1--4 (Weeks 1--8)}: Focus on U-Net model
    optimization and processing pipeline design
  \item \textbf{Sprints 5--8 (Weeks 9--12)}: Core component
    implementation and thread management
  \item \textbf{Sprints 9--12 (Weeks 13--16)}: Integration, testing,
    and optimization
\end{itemize}

\section{Risks and Risk Management}\label{sec:risk-management}

\subsection{Risk 1: Completion Delays}
\begin{itemize}
  \item \textbf{Probability}: 10\%
  \item \textbf{Severity}: High
  \item \textbf{Mitigation Strategies}:
    \begin{itemize}
      \item Regular sprint reviews to identify potential delays early
      \item Team members worked collaboratively on serialized tasks
        to avoid bottlenecks
      \item Maintained buffer time in the schedule for unexpected challenges
    \end{itemize}
\end{itemize}

\subsection{Risk 2: Hardware Damage}
\begin{itemize}
  \item \textbf{Probability}: 5\%
  \item \textbf{Severity}: Very High
  \item \textbf{Mitigation Strategies}:
    \begin{itemize}
      \item Store hardware in secure locations away from
        environmental contaminants
      \item Implement proper handling procedures for all team members
      \item Create regular backups of all work and configurations
    \end{itemize}
\end{itemize}

\subsection{Risk 3: Data Security}
\begin{itemize}
  \item \textbf{Probability}: 15\%
  \item \textbf{Severity}: Medium
  \item \textbf{Mitigation Strategies}:
    \begin{itemize}
      \item Utilize US-based distributed data storage (S3-compatible)
      \item Implement Git-based source and data version control
      \item Restrict access to sensitive data and systems
    \end{itemize}
\end{itemize}

\subsection{Risk 4: Algorithm Complexity}
\begin{itemize}
  \item \textbf{Probability}: 30\%
  \item \textbf{Severity}: Medium
  \item \textbf{Mitigation Strategies}:
    \begin{itemize}
      \item Implement modular design principles for better maintainability
      \item Conduct thorough code reviews to ensure clarity and efficiency
      \item Utilize comprehensive testing methodologies to validate integration
    \end{itemize}
\end{itemize}

\subsection{Risk 5: Pipeline and DPU Scheduling Implementation Challenges}
\begin{itemize}
  \item \textbf{Probability}: 40\%
  \item \textbf{Severity}: High
  \item \textbf{Mitigation Strategies}:
    \begin{itemize}
      \item Design efficient DPU scheduling system to maximize utilization
      \item Employ effective multithreading paradigms for CPU-based
        preprocessing and postprocessing
      \item Utilize synchronization primitives to coordinate between
        pipeline stages and avoid resource contention
      \item Profile and optimize critical code sections to maximize
        overall throughput
    \end{itemize}
\end{itemize}

\subsection{Risk 6: Image Processing Speed Limitations}
\begin{itemize}
  \item \textbf{Probability}: 25\%
  \item \textbf{Severity}: Medium
  \item \textbf{Mitigation Strategies}:
    \begin{itemize}
      \item Continuously optimize machine learning algorithms for
        semantic segmentation
      \item Implement data preprocessing optimizations
      \item Investigate model compression techniques to improve inference time
    \end{itemize}
\end{itemize}

For risks exceeding 30\% probability, we developed contingency plans
including alternative approaches and resource reallocation.

\section{Resource Requirements}\label{sec:resource-requirements}

\subsection{Hardware Resources}
\begin{itemize}
  \item AMD Kria KV260 development board~\cite{xilinx2022kv260}
  \item Compatible display devices for testing
  \item Storage for data backups and version control
  \item Network infrastructure for team collaboration
\end{itemize}

\subsection{Software Resources}
\begin{itemize}
  \item Vitis AI development suite
  \item Development tools (GCC, GDB, Make/CMake)
  \item ONNX runtime libraries
  \item Testing and benchmarking tools
  \item Cross-compilation toolchain for ARM targets
\end{itemize}

\subsection{Development Tools}
\begin{itemize}
  \item Version control (Git/GitHub)
  \item Communication platforms (Telegram)
  \item Documentation tools (LaTeX, Markdown editors)
  \item Performance profiling and analysis tools
\end{itemize}

\subsection{Data Resources}
\begin{itemize}
  \item Eye-tracking datasets for training and validation
  \item Ground truth data for accuracy measurement
  \item Performance benchmark datasets
  \item Medical episode data for distress detection validation
\end{itemize}
