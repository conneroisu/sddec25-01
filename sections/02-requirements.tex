\chapter{Requirements \& Standards}\label{chap:requirements}

\section{Requirements and Constraints}\label{sec:requirements-constraints}

\subsection{Functional Requirements}\label{subsec:functional-requirements}

\begin{enumerate}
  \item \textbf{Algorithm Optimization and Pipelining:}
    \begin{itemize}
      \item Optimize the U-Net semantic segmentation algorithm to
        enable efficient pipelined processing with CPU-based
        preprocessing and postprocessing stages.
      \item Implement a pipelined architecture that maximizes overall
        system throughput through overlapped CPU and DPU processing.
      \item Ensure the pipeline maintains data consistency and
        synchronization between stages with proper DPU scheduling
        across algorithms.
    \end{itemize}

  \item \textbf{System Throughput:}
    \begin{itemize}
      \item Achieve a system throughput of less than 35 ms total
        for processing four frames through the pipelined architecture
        (less than 10 ms per frame average).
      \item Ensure real-time processing capabilities are maintained
        for the assistive wheelchair application.
    \end{itemize}

  \item \textbf{Resource Efficiency:}
    \begin{itemize}
      \item Optimize memory and FPGA resource usage to accommodate
        the additional overhead of pipelined execution and
        multithreaded CPU processing.
      \item Ensure efficient sequential scheduling of the DPU among
        semantic segmentation and other algorithms.
    \end{itemize}

  \item \textbf{Error Handling in Pipeline:}
    \begin{itemize}
      \item Implement robust error handling mechanisms to detect and
        recover from pipeline stalls, frame drops, or data corruption.
    \end{itemize}
\end{enumerate}

\subsection{User Interface (UI) Requirements}\label{subsec:ui-requirements}

The demonstration user interface requirements provide clear visual
presentation of eye tracking and semantic segmentation in real-time.

\begin{enumerate}
  \item \textbf{Display Hardware:}
    \begin{itemize}
      \item The demonstration system shall utilize a VGA display
        connected directly to the Xilinx Kria KV260 development board
        for visual output.
      \item The VGA interface provides sufficient resolution and
        refresh rate to demonstrate real-time processing performance
        at 60 FPS\@.
    \end{itemize}

  \item \textbf{Desktop Environment:}
    \begin{itemize}
      \item The board shall run the XFCE desktop environment,
        providing a lightweight graphical user interface suitable for
        embedded systems.
      \item XFCE is deployed on a PetaLinux-based embedded Linux
        distribution~\cite{xilinx2022petalinux}, which provides the
        necessary drivers and system libraries for VGA output, camera
        interfacing, and neural network acceleration.
      \item The PetaLinux build shall be configured with appropriate
        kernel modules and device tree overlays to support the Kria
        KV260 hardware peripherals.
    \end{itemize}

  \item \textbf{Visual Demonstration:}
    \begin{itemize}
      \item The demonstration shall display real-time operation of
        the U-Net semantic segmentation model processing a live webcam feed.
      \item Eye tracking model results shall be overlaid on the video
        stream, clearly showing detected eye regions, pupil
        positions, and segmentation masks.
      \item The display shall include visual indicators of processing
        performance, such as frames per second (FPS) and inference
        latency metrics.
      \item The interface shall clearly demonstrate the pipelined
        architecture operation, showing synchronized processing of
        multiple frames through the CPU preprocessing, DPU inference,
        and CPU postprocessing stages.
    \end{itemize}

  \item \textbf{User Interaction:}
    \begin{itemize}
      \item Basic user controls shall be accessible through the XFCE
        environment, allowing demonstration operators to start, stop,
        and configure the eye tracking application.
      \item Configuration options shall include camera source
        selection, model parameters, and display overlay settings.
    \end{itemize}
\end{enumerate}

\subsection{Physical and Economic Requirements}\label{subsec:physical-economic}

\begin{enumerate}
  \item \textbf{Hardware Compatibility:}
    \begin{itemize}
      \item Ensure that the pipelined architecture remains compatible
        with the Xilinx Kria KV260 board.
      \item Minimize additional hardware requirements to keep costs low.
    \end{itemize}

  \item \textbf{Cost-Effectiveness:}
    \begin{itemize}
      \item Design the pipeline to maximize throughput without
        requiring significant hardware upgrades.
      \item Ensure that future maintenance and updates remain economical.
    \end{itemize}
\end{enumerate}

\subsection{System Constraints}\label{subsec:system-constraints}

\begin{enumerate}
  \item \textbf{Memory Limitations:}
    \begin{itemize}
      \item The Xilinx Kria K26 board has 4GB of DDR
        memory~\cite{xilinx2022kv260}, which must be shared among the
        pipeline stages.
      \item Optimize memory usage to avoid contention between stages
        and ensure smooth data flow~\cite{chen2022memory}.
    \end{itemize}

  \item \textbf{FPGA Resource Allocation:}
    \begin{itemize}
      \item The available FPGA resources are limited and must be
        efficiently allocated to accommodate the additional logic
        required for pipelining.
      \item Ensure that the DPU is shared effectively between blink
        detection and eye-tracking submodules.
    \end{itemize}

  \item \textbf{DPU Utilization:}
    \begin{itemize}
      \item Develop a sequential scheduling strategy for DPU access
        that efficiently coordinates between blink detection and
        eye-tracking submodules without causing bottlenecks.
    \end{itemize}
\end{enumerate}

\subsection{Client Constraints}\label{subsec:client-constraints}

The client imposed constraints that shaped the design and optimization approach:

\begin{enumerate}
  \item \textbf{FPGA Fabric Modification Prohibition:}
    \begin{itemize}
      \item The team cannot modify the FPGA fabric or DPU bitstream configuration.
      \item All optimizations must be implemented at the software level.
      \item The existing FPGA bitstream (which synthesizes the DPU onto the programmable logic) must
        remain unchanged throughout development.
    \end{itemize}

  \item \textbf{Single DPU Architecture Mandate:}
    \begin{itemize}
      \item The client requires maintaining the single B4096 DPU core (300MHz)
        configuration on the KV260 board.
      \item Despite analysis showing benefits of dual smaller DPUs (B1024 or B512),
        alternative configurations were not permitted.
      \item Both our team and a previous Iowa State team concluded dual smaller
        DPUs would be advantageous. We presented this finding with model size
        justifications, but the client directed maintaining the single B4096
        architecture.
      \item This constraint necessitates focus on efficient time-division
        scheduling and resource sharing rather than hardware-level parallelism.
    \end{itemize}

  \item \textbf{Accuracy Preservation Requirement:}
    \begin{itemize}
      \item The client requires maintaining 99.8\% IoU accuracy with no degradation.
      \item Given the medical nature of the product, any accuracy reduction is unacceptable.
      \item All optimizations must preserve eye tracking and monitoring reliability.
    \end{itemize}

  \item \textbf{Algorithm Resource Allocation:}
    \begin{itemize}
      \item Blink detection and eye tracking require periodic data collection;
        delayed collection produces incorrect information.
      \item Semantic segmentation cannot monopolize DPU resources or
        `starve' other critical algorithms of processing time.
      \item Fair scheduling mechanisms must ensure all algorithms receive
        appropriate DPU access.
    \end{itemize}

  \item \textbf{Existing System Integration:}
    \begin{itemize}
      \item Optimization must integrate with the existing architecture from previous teams.
      \item Backward compatibility with existing interfaces must be maintained.
      \item Medical applications require stability and proven reliability.
    \end{itemize}
\end{enumerate}

These constraints defined the optimization strategy boundaries. The FPGA
modification prohibition and single DPU mandate had the most impact,
requiring software-level optimization, efficient scheduling, and
multi-threaded CPU processing to achieve performance targets.

\subsection{Additional Considerations}\label{subsec:additional-considerations}

\begin{enumerate}
  \item \textbf{Deployment Options:}
    \begin{itemize}
      \item The system continues to be deployed on the Xilinx Kria
        KV260 board, with no immediate plans for expansion to other platforms.
      \item Ensure that the pipelined architecture is portable and
        can be adapted to future hardware upgrades if needed.
    \end{itemize}

  \item \textbf{Data Handling and Privacy:}
    \begin{itemize}
      \item Maintain strict data privacy and security measures,
        especially when handling sensitive user data in the pipeline.
      \item Ensure that intermediate data between pipeline stages is
        securely managed and not exposed to unauthorized access.
    \end{itemize}

  \item \textbf{Scalability:}
    \begin{itemize}
      \item Design the pipeline to be scalable, allowing for the
        addition of new algorithms or submodules in the future.
      \item Ensure that the architecture can handle increased
        workloads (e.g., higher frame rates or additional features)
        without significant rework.
    \end{itemize}
\end{enumerate}

\section{Engineering Standards}\label{sec:engineering-standards}

The following IEEE standards guide development and evaluation:

\begin{description}
  \item[IEEE 2802--2022~\cite{ieee2802-2022}] \textit{IEEE Standard
      for Performance and Safety Evaluation of AI-Based Medical
    Devices: Terminology} \\
    This standard defines terminology for evaluating AI-based medical
    devices. It guides our evaluation methodology for reliable real-time
    detection of medical episodes through eye tracking and posture analysis.

  \item[IEEE 3129--2023~\cite{ieee3129-2023}] \textit{IEEE Standard
      for Robustness Testing and Evaluation of AI-Based Image
    Recognition Services} \\
    This standard guides testing AI-based image recognition systems
    under varying conditions. Our U-Net model must maintain 99.8\%
    accuracy across different lighting, users, and environments.
    The standard informs our robustness testing in Chapter~\ref{chap:testing}.
\end{description}
