\chapter{Requirements \& Standards}\label{chap:requirements}

\section{Requirements and Constraints}\label{sec:requirements-constraints}

\subsection{Functional Requirements}\label{subsec:functional-requirements}

\begin{enumerate}
\item \textbf{Algorithm Optimization and Pipelining:}
   \begin{itemize}
   \item Optimize the U-Net semantic segmentation algorithm to enable efficient pipelined processing with CPU-based preprocessing and postprocessing stages.
   \item Implement a pipelined architecture that maximizes overall system throughput through overlapped CPU and DPU processing.
   \item Ensure the pipeline maintains data consistency and synchronization between stages with proper DPU scheduling across algorithms.
   \end{itemize}

\item \textbf{System Throughput:}
   \begin{itemize}
   \item Achieve a system throughput of less than 33.2 ms total for processing four frames through the pipelined architecture (approximately 8.3 ms per frame average).
   \item Ensure real-time processing capabilities are maintained for the assistive wheelchair application.
   \end{itemize}

\item \textbf{Resource Efficiency:}
   \begin{itemize}
   \item Optimize memory and FPGA resource usage to accommodate the additional overhead of pipelined execution and multithreaded CPU processing.
   \item Ensure efficient sequential scheduling of the DPU among semantic segmentation and other algorithms.
   \end{itemize}

\item \textbf{Error Handling in Pipeline:}
   \begin{itemize}
   \item Implement robust error handling mechanisms to detect and recover from pipeline stalls, frame drops, or data corruption.
   \end{itemize}
\end{enumerate}

\subsection{User Interface (UI) Requirements}\label{subsec:ui-requirements}

For the demonstration of the VisionAssist system, the user interface requirements are tailored to provide a clear and effective visual presentation of the eye tracking and semantic segmentation capabilities in real-time.

\begin{enumerate}
\item \textbf{Display Hardware:}
   \begin{itemize}
   \item The demonstration system shall utilize a VGA display connected directly to the Xilinx Kria KV260 development board for visual output.
   \item The VGA interface provides sufficient resolution and refresh rate to demonstrate real-time processing performance at 60 FPS\@.
   \end{itemize}

\item \textbf{Desktop Environment:}
   \begin{itemize}
   \item The board shall run the XFCE desktop environment, providing a lightweight graphical user interface suitable for embedded systems.
   \item XFCE is deployed on a PetaLinux-based embedded Linux distribution~\cite{xilinx2022petalinux}, which provides the necessary drivers and system libraries for VGA output, camera interfacing, and neural network acceleration.
   \item The PetaLinux build shall be configured with appropriate kernel modules and device tree overlays to support the Kria KV260 hardware peripherals.
   \end{itemize}

\item \textbf{Visual Demonstration:}
   \begin{itemize}
   \item The demonstration shall display real-time operation of the U-Net semantic segmentation model processing a live webcam feed.
   \item Eye tracking model results shall be overlaid on the video stream, clearly showing detected eye regions, pupil positions, and segmentation masks.
   \item The display shall include visual indicators of processing performance, such as frames per second (FPS) and inference latency metrics.
   \item The interface shall clearly demonstrate the pipelined architecture operation, showing synchronized processing of multiple frames through the CPU preprocessing, DPU inference, and CPU postprocessing stages.
   \end{itemize}

\item \textbf{User Interaction:}
   \begin{itemize}
   \item Basic user controls shall be accessible through the XFCE environment, allowing demonstration operators to start, stop, and configure the eye tracking application.
   \item Configuration options shall include camera source selection, model parameters, and display overlay settings.
   \end{itemize}
\end{enumerate}

\subsection{Physical and Economic Requirements}\label{subsec:physical-economic}

\begin{enumerate}
\item \textbf{Hardware Compatibility:}
   \begin{itemize}
   \item Ensure that the pipelined architecture remains compatible with the Xilinx Kria KV260 board.
   \item Minimize additional hardware requirements to keep costs low.
   \end{itemize}

\item \textbf{Cost-Effectiveness:}
   \begin{itemize}
   \item Design the pipeline to maximize throughput without requiring significant hardware upgrades.
   \item Ensure that future maintenance and updates remain economical.
   \end{itemize}
\end{enumerate}

\subsection{System Constraints}\label{subsec:system-constraints}

\begin{enumerate}
\item \textbf{Memory Limitations:}
   \begin{itemize}
   \item The Xilinx Kria K26 board has 4GB of DDR memory~\cite{xilinx2022kv260}, which must be shared among the pipeline stages.
   \item Optimize memory usage to avoid contention between stages and ensure smooth data flow~\cite{chen2022memory}.
   \end{itemize}

\item \textbf{FPGA Resource Allocation:}
   \begin{itemize}
   \item The available FPGA resources are limited and must be efficiently allocated to accommodate the additional logic required for pipelining.
   \item Ensure that the DPU is shared effectively between blink detection and eye-tracking submodules.
   \end{itemize}

\item \textbf{DPU Utilization:}
   \begin{itemize}
   \item Develop a sequential scheduling strategy for DPU access that efficiently coordinates between blink detection and eye-tracking submodules without causing bottlenecks.
   \end{itemize}
\end{enumerate}

\subsection{Client Constraints}\label{subsec:client-constraints}

The client imposed several critical constraints on the project that significantly shaped the design and optimization approach:

\begin{enumerate}
\item \textbf{FPGA Fabric Modification Prohibition:}
   \begin{itemize}
   \item The team was explicitly directed not to modify or touch the FPGA fabric containing the DPU\@.
   \item This constraint prevents hardware-level reconfiguration and requires all optimizations to be implemented at the software level.
   \item The existing FPGA programming and DPU integration must remain unchanged throughout development.
   \end{itemize}

\item \textbf{Single DPU Architecture Mandate:}
   \begin{itemize}
   \item The client specified maintaining the existing single B4096 DPU core (300MHz) configuration on the KV260 board.
   \item Despite technical analysis showing potential benefits of dual smaller DPUs (B1024 or B512), experimentation with alternative DPU configurations was not permitted.
   \item Both our team and a previous Iowa State team concluded that dual smaller DPUs would be advantageous for the workload. Our team formally presented this finding to the client with model size justifications and recommended the dual DPU approach as the optimal path forward. However, the client rejected this recommendation and directed the team to maintain the single B4096 architecture.
   \item This architectural constraint was established based on existing system design and integration considerations, necessitating focus on efficient time-division scheduling and resource sharing mechanisms rather than hardware-level parallelism.
   \end{itemize}

\item \textbf{Accuracy Preservation Requirement:}
   \begin{itemize}
   \item The client required maintaining the existing 99.8\% IoU accuracy with absolutely no degradation.
   \item Due to the sensitive medical nature of the product, any reduction in accuracy was unacceptable.
   \item All performance optimizations must preserve the precision and reliability of eye tracking and medical monitoring capabilities.
   \end{itemize}

\item \textbf{Algorithm Resource Allocation:}
   \begin{itemize}
   \item The client specified that blink detection and eye tracking algorithms require periodic data collection, and the information gathered would become incorrect if delayed.
   \item Semantic segmentation cannot monopolize DPU resources or `starve' other critical algorithms of processing time.
   \item This constraint necessitates implementation of fair scheduling mechanisms to ensure all algorithms receive appropriate DPU access.
   \end{itemize}

\item \textbf{Existing System Integration:}
   \begin{itemize}
   \item The optimization work must integrate with the existing system architecture developed by previous teams.
   \item Backward compatibility with existing interfaces and components must be maintained.
   \item The constraint reflects the reality of working with legacy embedded systems in medical applications where stability and proven reliability are paramount.
   \end{itemize}
\end{enumerate}

These client-imposed constraints collectively defined the boundaries within which the optimization strategy was developed. The prohibition on FPGA fabric modification and the single DPU architecture mandate had the most significant impact on the technical approach, forcing reliance on software-level optimization, efficient scheduling algorithms, and multi-threaded CPU processing to achieve performance targets.

\subsection{Additional Considerations}\label{subsec:additional-considerations}

\begin{enumerate}
\item \textbf{Deployment Options:}
   \begin{itemize}
   \item The system continues to be deployed on the Xilinx Kria KV260 board, with no immediate plans for expansion to other platforms.
   \item Ensure that the pipelined architecture is portable and can be adapted to future hardware upgrades if needed.
   \end{itemize}

\item \textbf{Data Handling and Privacy:}
   \begin{itemize}
   \item Maintain strict data privacy and security measures, especially when handling sensitive user data in the pipeline.
   \item Ensure that intermediate data between pipeline stages is securely managed and not exposed to unauthorized access.
   \end{itemize}

\item \textbf{Scalability:}
   \begin{itemize}
   \item Design the pipeline to be scalable, allowing for the addition of new algorithms or submodules in the future.
   \item Ensure that the architecture can handle increased workloads (e.g., higher frame rates or additional features) without significant rework.
   \end{itemize}
\end{enumerate}

\section{Engineering Standards}\label{sec:engineering-standards}

The following IEEE standards are applicable to this project and guide the development and evaluation processes:

\begin{description}
\item[IEEE 2802--2022~\cite{ieee2802-2022}] \textit{IEEE Standard for Performance and Safety Evaluation of AI-Based Medical Devices: Terminology} \\
This standard provides terminology and definitions for evaluating the performance and safety of AI-based medical devices. It is directly applicable to VisionAssist as an AI-powered assistive technology system for medical monitoring. The standard guides our evaluation methodology to ensure the system is reliable and effective in real-world medical settings, particularly for real-time detection of medical episodes through eye tracking and posture analysis.

\item[IEEE 3129--2023~\cite{ieee3129-2023}] \textit{IEEE Standard for Robustness Testing and Evaluation of AI-Based Image Recognition Services} \\
This standard provides guidelines for testing AI-based image recognition systems to ensure they operate reliably under varying conditions. It is directly applicable to our U-Net semantic segmentation model for eye tracking, which must maintain 99.8\% accuracy across different lighting conditions, user populations, and environmental factors. The standard informs our robustness testing methodology described in Chapter~\ref{chap:testing}, including stress testing, lighting variation testing, and long-term stability validation.
\end{description}
