\begin{abstract}
This research presents an innovative approach to optimizing semantic segmentation algorithms~\cite{garcia2017review,ronneberger2015} for real-time eye tracking in medical assistive technology applications. The proposed system implements efficient sequential DPU scheduling to ensure fair access to the Deep Processing Unit (DPU) across multiple algorithms, recognizing that the DPU can only execute one xmodel inference at a time. The pipelined architecture combines multi-threaded CPU processing with sequential DPU execution. Performance optimization targets a reduction from 160ms to approximately 8.3ms per frame average (33.2ms total for 4 frames in the pipeline) while maintaining~99.8\% Intersection over Union (IoU) accuracy. This optimization is critical for medical assistance devices serving individuals with mobility impairments, particularly those with cerebral palsy and similar conditions.

The system leverages the AMD Kria KV260 development board's multi-core ARM architecture and specialized single-DPU for sequential neural network inference. Key technical innovations include deadline-aware sequential DPU scheduling, memory management optimization for pipelined data flow, and CPU multi-threading strategies designed to maintain algorithmic accuracy while achieving 60 frames per second throughput. The architecture ensures that periodic data collection algorithms receive appropriate sequential DPU access time, preventing resource starvation that could compromise system reliability.

Initial development milestones include successful establishment of the development environment, validation of existing eye tracking algorithms, and implementation of a preliminary scheduling framework. Current performance metrics demonstrate 98.8\% IoU accuracy, indicating feasibility of reaching the target specifications. Subsequent development phases focus on resource management system refinement and inter-component data flow optimization.

This research directly enhances quality of life for individuals with disabilities by enabling more responsive and reliable eye-tracking interfaces. The improved processing capabilities facilitate faster detection and response to potential medical emergencies, providing caregivers and users with enhanced safety and autonomy in daily living activities.

\keywords{Semantic Segmentation; Eye Tracking; Assistive Technology; Real-Time Processing; U-Net; AMD Kria KV260; DPU Scheduling; Embedded Systems; Medical Device}
\end{abstract}
